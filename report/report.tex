\documentclass{article}
\usepackage{graphicx}
\usepackage[english,ukrainian]{babel}
\usepackage[letterpaper,top=2cm,bottom=2cm,left=3cm,right=3cm,marginparwidth=1.75cm]{geometry}
\usepackage{amsmath, graphicx, booktabs, listings, xcolor, tcolorbox, lipsum, siunitx, multirow, hyperref, pgfplots, inputenc}

\title{Пошук канонiчного розкладу великого числа, використовуючи вiдомi методи факторизацiї}
\date{}

\begin{document}

\maketitle

\section{Мета}
\quad Практичне ознайомлення з рiзними методами факторизацiї чисел, реалiзацiя цих методiв i їх порiвняння. Видiлення переваг, недолiкiв та особливостей застосування алгоритмiв факторизацiї. Застосування комбiнацiї алгоритмiв факторизацiї для пошуку канонiчного розкладу заданого числа.

\section{Постановка задачі}
\quad Створити та реалiзувати алгоритм для пошуку канонiчного розкладу числа.

\section{Хід роботи}
\quad 
Програма використовує кілька алгоритмів для знаходження канонічного розкладу числа. Спочатку перевіряється, чи є число простим за допомогою тесту Соловея-Штрассена, який використовує символ Якобі для визначення простоти числа з певною ймовірністю. Якщо число не є простим, використовується метод пробних ділень для знаходження дільників, які не перевищують 47. Якщо дільник не знайдено, застосовується метод Полларда для пошуку дільників шляхом випадкового перебору. У випадку, коли всі інші методи не знаходять дільник, застосовується метод Брілгарта-Моррісона.

\quad 
У нас виникди труднощі з реалізацією алгоритму Брілгарта-Моррісона, який для великих чисел працює дуже довго.

\section{Приклад роботи програми}
\quad 
Запустивши програму для числа 3009182572, маємо такий вивід: \\
Processing number: 3009182572 \\
Divisor found: 2, using method: Trial Division, at time: Thu Mar 21 20:40:01 2024 \\
Divisor found: 2, using method: Trial Division, at time: Thu Mar 21 20:40:01 2024 \\
Divisor found: 11, using method: Trial Division, at time: Thu Mar 21 20:40:01 2024 \\
Divisor found: 2063, using method: Pollard's Rho, at time: Thu Mar 21 20:40:01 2024 \\
Divisor found: 33151, using method: Solovay-Strassen (primality test), at time: Thu Mar 21 20:40:01 2024 \\
Algorithm execution time: 0.00661767s \\
Canonical expansion: 2 2 11 2063 33151  \\


\section{Висновок}
\quad
У цій роботі було розроблено програму для знаходження канонічного розкладу числа, використовуючи різноманітні методи факторизації, дана програма може бути вдосконалена, особливу увагу потрібно звернути на пошук гладки чисел для методу Брілгарта-Моррісона.
\end{document}
